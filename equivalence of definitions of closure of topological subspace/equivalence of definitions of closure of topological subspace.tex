\documentclass[12pt]{article}

\begin{filecontents}[overwrite]{mybib.bib}
@misc{Spronk.2019,
 author = {{Nico Spronk}},
 year   = {2019},
 title  = {Axiom of Choice et al},
 note   = {\url{http://www.math.uwaterloo.ca/~nspronk/math453/AofC.pdf}}
}
\end{filecontents}

%%% The following command loads the amsrefs package, which will be
%%% used to create the bibliography:
%\usepackage[lite]{amsrefs}

%%% The following command defines the standard names for all of the
%%% special symbols in the AMSfonts package, listed in
%%% http://www.ctan.org/tex-archive/info/symbols/math/symbols.pdf
\usepackage{amssymb}

%%% The following commands allow you to use \Xy-pic to draw
%%% commutative diagrams.  (You can omit the second line if you want
%%% the default style of the nodes to be \textstyle.)
\usepackage[all,cmtip]{xy}
\let\objectstyle=\displaystyle

\usepackage{hyperref}
\usepackage{mathtools}
\usepackage{url}

\usepackage{amsthm}
\let\proof\relax
\let\endproof\relax

\usepackage{lastpage}
\usepackage{fancyhdr}
\fancyfoot[C]{Page \thepage\ of \pageref*{LastPage}}
\pagestyle{fancy}

\usepackage{filemod}

\usepackage{natbib} % for \citep macro
\bibliographystyle{apalike}
\usepackage{xurl}

%\usepackage{enumitem,showframe}

%%% If you'll be importing any graphics, uncomment the following
%%% line.  (Note: The spelling is correct; the package graphicx.sty is
%%% the updated version of the older graphics.sty.)
% \usepackage{graphicx}

%%% This part of the file (after the \documentclass command,
%%% but before the \begin{document}) is called the ``preamble''.
%%% This is where we put our macro definitions.

%%% Comment out (or delete) any of these that you don't want to use.
\newcommand{\tensor}{\otimes}
\newcommand{\homotopic}{\simeq}
\newcommand{\homeq}{\cong}
\newcommand{\iso}{\approx}

\DeclareMathOperator{\ho}{Ho}
\DeclareMathOperator*{\colim}{colim}

\newcommand{\R}{\mathbb{R}}
\newcommand{\C}{\mathbb{C}}
\newcommand{\Z}{\mathbb{Z}}

\newcommand{\M}{\mathcal{M}}
\newcommand{\W}{\mathcal{W}}

\newcommand{\itilde}{\tilde{\imath}}
\newcommand{\jtilde}{\tilde{\jmath}}
\newcommand{\ihat}{\hat{\imath}}
\newcommand{\jhat}{\hat{\jmath}}

%\newcommand{\updateinfo}[1][\today]{\par\vfill\hfill{\script‌​size\color{gray}Last updated on #1}}

% The following causes equations to be numbered within sections
\numberwithin{equation}{section}

% We'll use the equation counter for all our theorem environments, so
% that everything will be numbered in the same sequence.

%       Theorem environments

\theoremstyle{plain} %% This is the default, anyway
\newtheorem{theorem}[equation]{Theorem}
%\newtheorem{theorem}{Theorem}[]
\newtheorem{cor}[equation]{Corollary}
\newtheorem{lemma}[equation]{Lemma}
\newtheorem{prop}[equation]{Proposition}
\newtheorem{statement}[equation]{Statement}

\theoremstyle{definition}
\newtheorem{definition}[equation]{Definition}
%\newtheorem{definition}{Definition}
%\newtheorem{proof}{{\bf Proof:}}{\hfill\rule{2mm}{2mm}}

\theoremstyle{remark}
\newtheorem{rem}[equation]{Remark}
\newtheorem{ex}[equation]{Example}
\newtheorem{notation}[equation]{Notation}
\newtheorem{terminology}[equation]{Terminology}

\newenvironment{proof}{{\bf Proof:}}{\hfill\rule{2mm}{2mm}}

%%%-------------------------------------------------------------------
%%%-------------------------------------------------------------------
%%%-------------------------------------------------------------------
%%%-------------------------------------------------------------------
%%%-------------------------------------------------------------------
%%%-------------------------------------------------------------------
%%%-------------------------------------------------------------------
\begin{document}

%%% In the title, use a double backslash "\\" to show a linebreak:
%%% Use one of the following two forms:
%%% \title{Text of the title}
%%% or
%%% \title[Short form for the running head]{Text of the title}
\title{Equivalence of definitions of closure of topological subspace}


%%% If there are multiple authors, they're described one at a time:
%%% First author: \author{} \address{} \curraddr{} \email{} \thanks{}
%%% Second author: \author{} \address{} \curraddr{} \email{} \thanks{}
%%% Third author: \author{} \address{} \curraddr{} \email{} \thanks{}
\author{Sibo WANG}

%%% In the address, show linebreaks with double backslashes:
%\address{}

%%% Current address is optional.
% \curraddr{}

%%% Email address is optional.
% \email{}


%%% If there's a second author:
% \author{}
% \address{}
% \curraddr{}
% \email{}


%%% To have the current date inserted, use \date{\today}:
\date{}

%%% To include an abstract, uncomment the following two lines and type
%%% the abstract in between them:
% \begin{abstract}
% \end{abstract}


\maketitle

%%% To include a table of contents, uncomment the following line:
% \tableofcontents


%%%-------------------------------------------------------------------
%%%-------------------------------------------------------------------
%%% Start the body of the paper here!  E.G., maybe use:
%%% \section{Introduction}
%%% \label{sec:intro}

%%% For a numbered display, use
%%% \begin{equation}
%%%   \label{something}
%%%   The display goes here
%%% \end{equation}
%%% and you can refer to it as \eqref{something}.

%%% For an unnumbered display, use
%%% \begin{equation*}
%%%   The display goes here
%%% \end{equation*}

%%% To import a graphics file, you must have said
%%% \usepackage{graphicx}
%%% in the preamble (i.e., before the \begin{document}).
%%% Putting it into a figure environment enables it to float to the
%%% next page if there isn't enough room for it on the current page.
%%% The \label command must come after the \caption command.
% \begin{figure}[h]
%   \includegraphics{filename}
%   \caption{Some caption}
%   \label{somelabel}
% \end{figure}

\section{Motivation}
%\label{sec:intro}
\textit{Closure} in topology is an essential concept and it has various definitions. In this note, equivalence of those definitions will be proved.

\section{Equivalence of definitions of closure of topological subspace}
%\label{sec:intro}

\begin{lemma}
Given a topology space $(X, \mathcal{T})$. Infinite intersection of closed set in $X$ is a closed set. \label{lemma:intersection of closed set is closed set}
\end{lemma}

\begin{proof}
(De Morgan's laws)
\end{proof}

\begin{definition}
Given a topology space $(X, \mathcal{T})$, for $A \subseteq X$, a point $x \in X$ is called \textit{a limit point of A} if for each $B \in \{B: x \in B \in \mathcal{T} \}, B \cap A \neq \varnothing$. \label{definition:limit point}
\end{definition}

\begin{lemma}
Given a topology space $(X, \mathcal{T})$ and $A \subseteq X$. If $A'$ is the set that only contains all limit points of $A$, then $A \cup A'$ is closed.  \label{lemma:limit point set is closed}
\end{lemma}

\begin{proof}
To be continued
\end{proof}

\begin{notation}
$\overline{A}$: closure of $A$ in a topology space $(X, \mathcal{T})$. 
\end{notation}

\begin{theorem}
	The following definitions are equivalent:
	\begin{enumerate}
		\item  Let $A'$ denote the set of all limit points of $A$, then $\overline{A} = A \bigcup A'$. \label{definition:closure 1}
		\item $\overline{A} = \bigcap \{A \subseteq B: \text{B is closed in }X\}$ \label{definition:closure 2}
		\item $\overline{A}$ is the smallest closed set containing $A$.   
	\end{enumerate}
\end{theorem}

\begin{proof}
\begin{description}
	\item[$1 \Rightarrow 2$:] On the one hand, given $A$, for every $B \in \{A \subseteq B: \text{B is closed in }X\}$, $\overline{A} \subseteq \overline{B}$. Because $B$ is closed, $\overline{A} \subseteq {B}$ by \hyperref[lemma:topological closure of subset is subset of topological closure]{lemma \ref*{lemma:topological closure of subset is subset of topological closure}}. Thus $\overline{A} \subseteq \bigcap \{A \subseteq B: \text{B is closed in }X\}$. \\
	On the other hand, $\overline{A}$ is a closed set containing $A$ by \hyperref[lemma:limit point set is closed]{lemma \ref*{lemma:limit point set is closed}}, so $\bigcap \{A \subseteq B: \text{B is closed in }X\} \subseteq \overline{A}$ by \hyperref[definition:closure 2]{2}. \\
	Therefore, $\overline{A} = \bigcap \{A \subseteq B: \text{B is closed in }X\}$. 
	\item[$2 \Rightarrow 1$:] On the one hand, $\overline{A}$ is a closed set containing $A$ by \hyperref[lemma:limit point set is closed]{lemma \ref*{lemma:limit point set is closed}}, so $\bigcap \{A \subseteq B: \text{B is closed in }X\} \subseteq \overline{A}$ by \hyperref[definition:closure 2]{2}. \\
	On the other hand, proof by contradiction. We assume $\bigcap \{A \subseteq B: \text{B is closed in }X\} \subset \overline{A}$, then by \hyperref[lemma:intersection of closed set is closed set]{lemma \ref*{lemma:intersection of closed set is closed set}} there exists a closed set $B \in \{A \subseteq B: \text{B is closed in }X\}$ such that $A \subseteq B \subset \overline{A}$. Let $C$ denote $\overline{A} \setminus B$, so for each $c \in C$, they are called a limit point of A. Then, there exists an open set $X \setminus B$ such that $c \in X \setminus B$ for each $c \in C$ and $(X \setminus B) \cap A = \varnothing$, which violates \hyperref[definition:limit point]{definition \ref*{definition:limit point}}. Thus the assumption is wrong and $\bigcap \{A \subseteq B: \text{B is closed in }X\} \supseteq \overline{A}$.\\
	Therefore, $\overline{A} = \bigcap \{A \subseteq B: \text{B is closed in }X\}$. 
	\item[$2 \Rightarrow 3$:] \hyperref[lemma:intersection of closed set is closed set]{lemma \ref*{lemma:intersection of closed set is closed set}} 
	\item[$3 \Rightarrow 2$:] \hyperref[lemma:intersection of closed set is closed set]{lemma \ref*{lemma:intersection of closed set is closed set}} 
\end{description}
\end{proof}

\begin{lemma}
Given \hyperref[definition:closure 1]{1}, if $A \subseteq B$ in a topology space $(X, \mathcal{T})$, then $\overline{A} \subseteq \overline{B}$. \label{lemma:topological closure of subset is subset of topological closure} \label{lemma:topological closure of subset is subset of topological closure}
\end{lemma}

\begin{proof}
(using \hyperref[lemma:limit point set is closed]{lemma \ref*{lemma:limit point set is closed}}, proof by contradiction)
\end{proof}

%\begin{lemma}
%Given a topology space $(X, \mathcal{T})$, if $A \subseteq B \subseteq X$, then $\overline{A} \subseteq \overline{B}$. 
%\end{lemma}
%
%\begin{proof}
%To be continued
%\end{proof}

%%% -------------------------------------------------------------------
%%% -------------------------------------------------------------------
%%% This is where we create the bibliography.

%\begin{bibdiv}
%  	\begin{biblist}
%  	
%%\url(http://www.math.uwaterloo.ca/~nspronk/math453/AofC.pdf)\\
%%\url(http://www.math.toronto.edu/ivan/mat327/docs/notes/11-choice.pdf)
%
%
%	\end{biblist}
%\end{bibdiv}

%\bibliography{mybib}

%\printbibliography


\end{document}