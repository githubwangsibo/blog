\documentclass[12pt]{article}

\begin{filecontents}[overwrite]{mybib.bib}
\end{filecontents}

%%% The following command loads the amsrefs package, which will be
%%% used to create the bibliography:
%\usepackage[lite]{amsrefs}

%%% The following command defines the standard names for all of the
%%% special symbols in the AMSfonts package, listed in
%%% http://www.ctan.org/tex-archive/info/symbols/math/symbols.pdf
\usepackage{amssymb}

%%% The following commands allow you to use \Xy-pic to draw
%%% commutative diagrams.  (You can omit the second line if you want
%%% the default style of the nodes to be \textstyle.)
\usepackage[all,cmtip]{xy}
\let\objectstyle=\displaystyle

\usepackage{hyperref}
\hypersetup{
%    bookmarks=true,         % show bookmarks bar?
%    unicode=false,          % non-Latin characters in Acrobat’s bookmarks
%    pdftoolbar=true,        % show Acrobat’s toolbar?
%    pdfmenubar=true,        % show Acrobat’s menu?
%    pdffitwindow=false,     % window fit to page when opened
%    pdfstartview={FitH},    % fits the width of the page to the window
%    pdftitle={My title},    % title
%    pdfauthor={Author},     % author
%    pdfsubject={Subject},   % subject of the document
%    pdfcreator={Creator},   % creator of the document
%    pdfproducer={Producer}, % producer of the document
%    pdfkeywords={keyword1, key2, key3}, % list of keywords
%    pdfnewwindow=true,      % links in new PDF window
    colorlinks=true,       % false: boxed links; true: colored links
%    linkcolor=red,          % color of internal links (change box color with linkbordercolor)
%    citecolor=green,        % color of links to bibliography
%    filecolor=cyan,         % color of file links
%    urlcolor=magenta        % color of external links
}
\usepackage{mathtools}
\usepackage{url}

\usepackage{amsthm}
\let\proof\relax
\let\endproof\relax

\usepackage{lastpage}
\usepackage{fancyhdr}
\fancyfoot[C]{Page \thepage\ of \pageref*{LastPage}}
\pagestyle{fancy}

\usepackage{filemod}

\usepackage{natbib} % for \citep macro
\bibliographystyle{apalike}
\usepackage{xurl}

%\usepackage{enumitem,showframe}

%%% If you'll be importing any graphics, uncomment the following
%%% line.  (Note: The spelling is correct; the package graphicx.sty is
%%% the updated version of the older graphics.sty.)
% \usepackage{graphicx}

%%% This part of the file (after the \documentclass command,
%%% but before the \begin{document}) is called the ``preamble''.
%%% This is where we put our macro definitions.

%%% Comment out (or delete) any of these that you don't want to use.
\newcommand{\tensor}{\otimes}
\newcommand{\homotopic}{\simeq}
\newcommand{\homeq}{\cong}
\newcommand{\iso}{\approx}

\DeclareMathOperator{\ho}{Ho}
\DeclareMathOperator*{\colim}{colim}

\newcommand{\R}{\mathbb{R}}
\newcommand{\C}{\mathbb{C}}
\newcommand{\Z}{\mathbb{Z}}

\newcommand{\M}{\mathcal{M}}
\newcommand{\W}{\mathcal{W}}

\newcommand{\itilde}{\tilde{\imath}}
\newcommand{\jtilde}{\tilde{\jmath}}
\newcommand{\ihat}{\hat{\imath}}
\newcommand{\jhat}{\hat{\jmath}}

%\newcommand{\updateinfo}[1][\today]{\par\vfill\hfill{\script‌​size\color{gray}Last updated on #1}}

% The following causes equations to be numbered within sections
\numberwithin{equation}{section}

% We'll use the equation counter for all our theorem environments, so
% that everything will be numbered in the same sequence.

%       Theorem environments

\theoremstyle{plain} %% This is the default, anyway
\newtheorem{theorem}[equation]{Theorem}
%\newtheorem{theorem}{Theorem}[]
\newtheorem{cor}[equation]{Corollary}
\newtheorem{lemma}[equation]{Lemma}
\newtheorem{prop}[equation]{Proposition}
\newtheorem{statement}[equation]{Statement}

\theoremstyle{definition}
\newtheorem{definition}[equation]{Definition}
%\newtheorem{definition}{Definition}
%\newtheorem{proof}{{\bf Proof:}}{\hfill\rule{2mm}{2mm}}
\newtheorem{example}[equation]{Example}

\theoremstyle{remark}
\newtheorem{rem}[equation]{Remark}
%\newtheorem{example}[equation]{Example}
\newtheorem{notation}[equation]{Notation}
\newtheorem{terminology}[equation]{Terminology}

\newenvironment{proof}{{\bf Proof:}}{\hfill\rule{2mm}{2mm}}
\newenvironment{solution}{{\bf Solution:}}{\hfill\rule{2mm}{2mm}}

%%%-------------------------------------------------------------------
%%%-------------------------------------------------------------------
%%%-------------------------------------------------------------------
%%%-------------------------------------------------------------------
%%%-------------------------------------------------------------------
%%%-------------------------------------------------------------------
%%%-------------------------------------------------------------------
\begin{document}

%%% In the title, use a double backslash "\\" to show a linebreak:
%%% Use one of the following two forms:
%%% \title{Text of the title}
%%% or
%%% \title[Short form for the running head]{Text of the title}
\title{Infinite Series}


%%% If there are multiple authors, they're described one at a time:
%%% First author: \author{} \address{} \curraddr{} \email{} \thanks{}
%%% Second author: \author{} \address{} \curraddr{} \email{} \thanks{}
%%% Third author: \author{} \address{} \curraddr{} \email{} \thanks{}
\author{Sibo WANG}

%%% In the address, show linebreaks with double backslashes:
%\address{}

%%% Current address is optional.
% \curraddr{}

%%% Email address is optional.
% \email{}


%%% If there's a second author:
% \author{}
% \address{}
% \curraddr{}
% \email{}


%%% To have the current date inserted, use \date{\today}:
\date{}

%%% To include an abstract, uncomment the following two lines and type
%%% the abstract in between them:
% \begin{abstract}
% \end{abstract}


\maketitle

%%% To include a table of contents, uncomment the following line:
% \tableofcontents


%%%-------------------------------------------------------------------
%%%-------------------------------------------------------------------
%%% Start the body of the paper here!  E.G., maybe use:
%%% \section{Introduction}
%%% \label{sec:intro}

%%% For a numbered display, use
%%% \begin{equation}
%%%   \label{something}
%%%   The display goes here
%%% \end{equation}
%%% and you can refer to it as \eqref{something}.

%%% For an unnumbered display, use
%%% \begin{equation*}
%%%   The display goes here
%%% \end{equation*}

%%% To import a graphics file, you must have said
%%% \usepackage{graphicx}
%%% in the preamble (i.e., before the \begin{document}).
%%% Putting it into a figure environment enables it to float to the
%%% next page if there isn't enough room for it on the current page.
%%% The \label command must come after the \caption command.
% \begin{figure}[h]
%   \includegraphics{filename}
%   \caption{Some caption}
%   \label{somelabel}
% \end{figure}

\section{Introduction}
%\label{sec:intro}
Infinite series is widely used in almost all aspects of mathematics. Several technique in infinite series with examples will be discussed.  

\section{Example}
%\label{sec:intro}

\begin{example}
Given a sequence \{$x_n$\} with an upper bound, $x_{n+1} > x_{n} - o(\frac{1}{n})$ for all $n \in \mathbb{N}$, show that \{$x_n$\} is convergent. 
\end{example}

\begin{proof}
Since \{$x_n$\} has an upper bound and \href{https://en.wikipedia.org/wiki/Least-upper-bound_property}{least upper bound property}, the supremum of \{$x_n$\} exists. We denote $U$ is supremum of \{$x_n$\}
$$U = \inf\limits_{k \in \mathbb{N}}\sup\limits_{n \geqslant k}\{x_n\}$$
We construct a new sequence $\{y_n\}$ 
$$\{y_n\} = \{y_k \mid y_k = \sup\limits_{n \geqslant k}\{x_n\} \text{ for all }k \in \mathbb{N}\}$$ 
It is obvious that $y_n \geqslant x_n$ and $y_n \geqslant U$ for all $k \in \mathbb{N}$. \\
In addition, $\{y_n\}$ is convergent to $U$. In another word, for each $\varepsilon > 0$, there exists $N_1(\varepsilon)$ such that $0 \leqslant y_n - U < \varepsilon$ for all $n > N_1(\varepsilon)$.\\
Therefore, for each $\varepsilon > 0$, there exists $N_1(\varepsilon)$ such that $x_n - U < \varepsilon$ for all $n > N_1(\varepsilon)$. \\
Since $\sum_{n=1}^{\infty} \frac{1}{n^s}$ is convergent with Re($s$) $>$ 1, we have $$x_n - x_m < O(\frac{1}{n})\text{ for all }m > n$$
Because $U$ is supremum of \{$x_n$\}, we might have two conditions: 
\begin{enumerate}
	\item If for every $n \in \mathbb{N}$, there exists $m > n$ such that $x_m \geqslant U$. Then for each $\varepsilon > 0$, there exists $n = N_2(\varepsilon) \in \mathbb{N}$ such that $\frac{C}{N_2(\varepsilon)} < \varepsilon$ with constant $C$ and $0 \leqslant x_n - U < \varepsilon$.\\
	Then for each $\varepsilon > 0$, there exists $n = N_2(\varepsilon)$ such that $U - x_m \leqslant x_n - x_m < O(\frac{1}{n}) < \frac{C}{N_2(\varepsilon)} < \varepsilon$ for all $m > n$. \\
	Therefore, for each $\varepsilon > 0$, there exists $N_1(\varepsilon)$ and $N_2(\varepsilon)$ such that $\lvert x_n - U \rvert< \varepsilon$ for all $n > \max\{N_1(\varepsilon), N_2(\varepsilon)\}$.
	
	\item If there exists $L \in \mathbb{N}$ such that $x_k < U$ for all $k \geqslant L$. Then we have an infinite sub-sequence \{$z_n$\}
	$$\{z_n\} = \{x_k \mid x_k \in \{x_n\}\text{ and }k \geqslant L \}$$ 
	and
	$$\inf\limits_{k \in \mathbb{N}}\sup\limits_{n \geqslant k}\{z_n\} = \inf\limits_{k \in \mathbb{N}}\sup\limits_{n \geqslant k}\{x_n\} = U$$
	Thus for each $\varepsilon > 0$, there exists $n = N_3(\varepsilon)$ such that $U - z_n < \varepsilon$ and $z_n - z_m < O(\frac{1}{n}) \leqslant \frac{C}{n}$ with constant $C$ for all $m > n$. \\
	Therefore for each $\varepsilon > 0$, there exists $n = \max\{\lceil \frac{2C}{\varepsilon} \rceil, N_3(\frac{\varepsilon}{2})\}$ such that $C$ is constant and $U - z_m = U - z_n + z_n - z_m < \frac{\varepsilon}{2} + \frac{\varepsilon}{2} = \varepsilon$ for all $m > n$. 
	Therefore for each $\varepsilon > 0$, there exists $n = \max\{\lceil \frac{2C}{\varepsilon} \rceil + L, N_3(\frac{\varepsilon}{2}) + L\}$ such that $C$ is constant and $\lvert U - x_m \rvert < \varepsilon$ for all $m > n$. 
\end{enumerate}

\end{proof}

%%% -------------------------------------------------------------------
%%% -------------------------------------------------------------------
%%% This is where we create the bibliography.

%\begin{bibdiv}
%  	\begin{biblist}
%  	
%
%
%	\end{biblist}
%\end{bibdiv}

%\bibliography{mybib}

%\printbibliography


\end{document}