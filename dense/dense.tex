\documentclass[12pt]{article}

\begin{filecontents}[overwrite]{mybib.bib}
\end{filecontents}

%%% The following command loads the amsrefs package, which will be
%%% used to create the bibliography:
%\usepackage[lite]{amsrefs}

%%% The following command defines the standard names for all of the
%%% special symbols in the AMSfonts package, listed in
%%% http://www.ctan.org/tex-archive/info/symbols/math/symbols.pdf
\usepackage{amssymb}

%%% The following commands allow you to use \Xy-pic to draw
%%% commutative diagrams.  (You can omit the second line if you want
%%% the default style of the nodes to be \textstyle.)
\usepackage[all,cmtip]{xy}
\let\objectstyle=\displaystyle

\usepackage{hyperref}
\hypersetup{
%    bookmarks=true,         % show bookmarks bar?
%    unicode=false,          % non-Latin characters in Acrobat’s bookmarks
%    pdftoolbar=true,        % show Acrobat’s toolbar?
%    pdfmenubar=true,        % show Acrobat’s menu?
%    pdffitwindow=false,     % window fit to page when opened
%    pdfstartview={FitH},    % fits the width of the page to the window
%    pdftitle={My title},    % title
%    pdfauthor={Author},     % author
%    pdfsubject={Subject},   % subject of the document
%    pdfcreator={Creator},   % creator of the document
%    pdfproducer={Producer}, % producer of the document
%    pdfkeywords={keyword1, key2, key3}, % list of keywords
%    pdfnewwindow=true,      % links in new PDF window
    colorlinks=true,       % false: boxed links; true: colored links
%    linkcolor=red,          % color of internal links (change box color with linkbordercolor)
%    citecolor=green,        % color of links to bibliography
%    filecolor=cyan,         % color of file links
%    urlcolor=magenta        % color of external links
}
\usepackage{mathtools}
\usepackage{url}

\usepackage{amsthm}
\let\proof\relax
\let\endproof\relax

\usepackage{lastpage}
\usepackage{fancyhdr}
\fancyfoot[C]{Page \thepage\ of \pageref*{LastPage}}
\pagestyle{fancy}

\usepackage{filemod}

\usepackage{natbib} % for \citep macro
\bibliographystyle{apalike}
\usepackage{xurl}

%\usepackage{enumitem,showframe}

%%% If you'll be importing any graphics, uncomment the following
%%% line.  (Note: The spelling is correct; the package graphicx.sty is
%%% the updated version of the older graphics.sty.)
% \usepackage{graphicx}

%%% This part of the file (after the \documentclass command,
%%% but before the \begin{document}) is called the ``preamble''.
%%% This is where we put our macro definitions.

%%% Comment out (or delete) any of these that you don't want to use.
\newcommand{\tensor}{\otimes}
\newcommand{\homotopic}{\simeq}
\newcommand{\homeq}{\cong}
\newcommand{\iso}{\approx}

\DeclareMathOperator{\ho}{Ho}
\DeclareMathOperator*{\colim}{colim}

\newcommand{\R}{\mathbb{R}}
\newcommand{\C}{\mathbb{C}}
\newcommand{\Z}{\mathbb{Z}}

\newcommand{\M}{\mathcal{M}}
\newcommand{\W}{\mathcal{W}}

\newcommand{\itilde}{\tilde{\imath}}
\newcommand{\jtilde}{\tilde{\jmath}}
\newcommand{\ihat}{\hat{\imath}}
\newcommand{\jhat}{\hat{\jmath}}

%\newcommand{\updateinfo}[1][\today]{\par\vfill\hfill{\script‌​size\color{gray}Last updated on #1}}

% The following causes equations to be numbered within sections
\numberwithin{equation}{section}

% We'll use the equation counter for all our theorem environments, so
% that everything will be numbered in the same sequence.

%       Theorem environments

\theoremstyle{plain} %% This is the default, anyway
\newtheorem{theorem}[equation]{Theorem}
%\newtheorem{theorem}{Theorem}[]
\newtheorem{cor}[equation]{Corollary}
\newtheorem{lemma}[equation]{Lemma}
\newtheorem{prop}[equation]{Proposition}
\newtheorem{statement}[equation]{Statement}

\theoremstyle{definition}
\newtheorem{definition}[equation]{Definition}
%\newtheorem{definition}{Definition}
%\newtheorem{proof}{{\bf Proof:}}{\hfill\rule{2mm}{2mm}}
\newtheorem{example}[equation]{Example}
\newtheorem{warning}[equation]{Warning}

\theoremstyle{remark}
\newtheorem{rem}[equation]{Remark}
%\newtheorem{example}[equation]{Example}
\newtheorem{notation}[equation]{Notation}
\newtheorem{terminology}[equation]{Terminology}

\newenvironment{proof}{{\bf Proof:}}{\hfill\rule{2mm}{2mm}}
\newenvironment{solution}{{\bf Solution:}}{\hfill\rule{2mm}{2mm}}

%%%-------------------------------------------------------------------
%%%-------------------------------------------------------------------
%%%-------------------------------------------------------------------
%%%-------------------------------------------------------------------
%%%-------------------------------------------------------------------
%%%-------------------------------------------------------------------
%%%-------------------------------------------------------------------
\begin{document}

%%% In the title, use a double backslash "\\" to show a linebreak:
%%% Use one of the following two forms:
%%% \title{Text of the title}
%%% or
%%% \title[Short form for the running head]{Text of the title}
\title{Dense}


%%% If there are multiple authors, they're described one at a time:
%%% First author: \author{} \address{} \curraddr{} \email{} \thanks{}
%%% Second author: \author{} \address{} \curraddr{} \email{} \thanks{}
%%% Third author: \author{} \address{} \curraddr{} \email{} \thanks{}
\author{Sibo WANG}

%%% In the address, show linebreaks with double backslashes:
%\address{}

%%% Current address is optional.
% \curraddr{}

%%% Email address is optional.
% \email{}


%%% If there's a second author:
% \author{}
% \address{}
% \curraddr{}
% \email{}


%%% To have the current date inserted, use \date{\today}:
\date{}

%%% To include an abstract, uncomment the following two lines and type
%%% the abstract in between them:
% \begin{abstract}
% \end{abstract}


\maketitle

%%% To include a table of contents, uncomment the following line:
% \tableofcontents


%%%-------------------------------------------------------------------
%%%-------------------------------------------------------------------
%%% Start the body of the paper here!  E.G., maybe use:
%%% \section{Introduction}
%%% \label{sec:intro}

%%% For a numbered display, use
%%% \begin{equation}
%%%   \label{something}
%%%   The display goes here
%%% \end{equation}
%%% and you can refer to it as \eqref{something}.

%%% For an unnumbered display, use
%%% \begin{equation*}
%%%   The display goes here
%%% \end{equation*}

%%% To import a graphics file, you must have said
%%% \usepackage{graphicx}
%%% in the preamble (i.e., before the \begin{document}).
%%% Putting it into a figure environment enables it to float to the
%%% next page if there isn't enough room for it on the current page.
%%% The \label command must come after the \caption command.
% \begin{figure}[h]
%   \includegraphics{filename}
%   \caption{Some caption}
%   \label{somelabel}
% \end{figure}

\section{Motivation}
%\label{sec:intro}
\textit{Dense} is a crucial concept in topology and analysis. Several related theorems will be discussed. 

\section{Separable}
\begin{theorem}
If a normed vector space has a \textit{schauder basis}, then it is \textit{separable}. 
\end{theorem}

\begin{proof}
Given a normed vector space ($X$, $\lVert \cdot \rVert_{X}$) on field $\mathbb{F}$, we have to construct a subtle subspace ($Y$, $\lVert \cdot \rVert_{X}$) and show that $Y \subset X$  is countable and dense. \\
We use $\{e_n\}$ to denote the schauder basis of ($X$, $\lVert \cdot \rVert_{X}$). 
It is true that there exists a dense and countable subset $\mathbb{G} \subset \mathbb{F}$. \\
Given basis $\{e_n\}$, we define the subset $Y$
$$Y = \bigcup_{n \in \mathbb{N}}\{y_n \mid y_n = \sum_{k=1}^{n} g_k e_k \text{ and } g_k \in \mathbb{G}\}$$
The function $\bigcup_{n \in \mathbb{N}} G^{n} \to Y$ is bijective and $G^{n}$ is countable for all $n \in \mathbb{N}$. Thus $Y$ is countable. \\
Next, it suffices to show $Y$ is dense. \\
Due to schauder basis $\{e_n\}$, for each $x \in X$ and $\varepsilon > 0$, there exist $N_1(\varepsilon)$ and $\{f_k(\varepsilon)\}$ such that $\lVert x - \sum_{k=1}^{n} f_k e_k \rVert_{X} < \varepsilon$ and $f_k \in \mathbb{F}$ if $n > N_1(\varepsilon)$ for n. \\
Moreover, because $\mathbb{G}$ is dense, for each $f_k \in \mathbb{F}$, $k \in \mathbb{N}$ and $\varepsilon > 0$, there exists $g_k(\varepsilon) \in \mathbb{G}$ such that $\lvert f_k - g_k \rvert < \varepsilon$. \\
Given $x$ and $\{e_n\}$, for each $\varepsilon > 0$, there exist $N_1(\frac{\varepsilon}{2})$, $\{\frac{\varepsilon}{2N_1(\frac{\varepsilon}{2})\lVert e_k \rVert_{X}}\}$ and $\{f_k(\frac{\varepsilon}{2})\}$ such that
\begin{equation*}
\begin{split}
\lVert x - \sum_{k=1}^{n} g_k e_k \rVert_{X}
	& = \lVert x - \sum_{k=1}^{n} f_k e_k + \sum_{k=1}^{n} f_k e_k - \sum_{k=1}^{n} g_k e_k \rVert_{X} \\
	& \leqslant \lVert x - \sum_{k=1}^{n} f_k e_k \rVert_{X} + \lVert \sum_{k=1}^{n} f_k e_k - \sum_{k=1}^{n} g_k e_k \rVert_{X}\\
	& < \frac{\varepsilon}{2} + \sum_{k=1}^{n} \lvert f_k - g_k \rvert \lVert e_k \rVert_{X}\\
	& < \frac{\varepsilon}{2} + \frac{\varepsilon}{2}\\
	&= \varepsilon
\end{split}
\end{equation*}
if $n > N_1(\frac{\varepsilon}{2})$ and $\lvert f_k - g_k \rvert < \frac{\varepsilon}{2N_1(\frac{\varepsilon}{2})\lVert e_k \rVert_{X}}$ for $n$ and $\{g_k\}$. \\
Therefore $Y$ is a dense and countable subset and ($X$, $\lVert \cdot \rVert_{X}$) is separable. 
\end{proof}

\begin{warning}
The above normed vector space ($X$, $\lVert \cdot \rVert_{X}$) is not necessary to be \textit{complete}. 
\end{warning}

\begin{example}
Given a normed vector space ($X$, $\lVert \cdot \rVert_{X}$), $X = C^{0}[0, 1]$ and $\lVert f \rVert_{X} = \lVert f \rVert_{2} = {(\int_{0}^{1} {f(t)}^{2} \, dt)}^{\frac{1}{2}}$. Then ($X$, $\lVert \cdot \rVert_{X}$) is not complete. 
\end{example}

\begin{proof}
We have the sequence $\{f_n\} \subset X$
\begin{equation*}
    f_n(x) = 
    \begin{cases*}
		0  & $x \in [0, \frac{1}{2} - \frac{1}{2^{n}})$ \\
		(x - \frac{1}{2})2^{n-1} + \frac{1}{2}      & $x \in [\frac{1}{2} - \frac{1}{2^{n}}, \frac{1}{2} + \frac{1}{2^{n}}]$ \\
     	1       & $x \in (\frac{1}{2^{n}} + 2^{-n}, 1]$
    \end{cases*}
\end{equation*}
It is obvious that $\{f_n\}$ is convergent to $f$
\begin{equation*}
    f(x) = 
    \begin{cases*}
		0  & $x \in [0, \frac{1}{2})$ \\
		\frac{1}{2}      & $x = \frac{1}{2}$ \\
     	1       & $x \in (\frac{1}{2}, 1]$
    \end{cases*}
\end{equation*}
and $f \notin X$. 
\end{proof}

%%% -------------------------------------------------------------------
%%% -------------------------------------------------------------------
%%% This is where we create the bibliography.

%\begin{bibdiv}
%  	\begin{biblist}
%  	
%
%
%	\end{biblist}
%\end{bibdiv}

%\bibliography{mybib}

%\printbibliography


\end{document}