%%% template.tex
%%% This is a template for making up an AMS-LaTeX file
%%% Version of February 12, 2011
%%%---------------------------------------------------------
%%% The following command chooses the default 10 point type.
%%% To choose 12 point, change it to
%\documentclass[12pt]{amsart}
\documentclass{article}

\begin{filecontents}[overwrite]{mybib.bib}
@misc{Spronk.2019,
 author = {{Nico Spronk}},
 year   = {2019},
 title  = {Axiom of Choice et al},
 note   = {\url{http://www.math.uwaterloo.ca/~nspronk/math453/AofC.pdf}}
}
\end{filecontents}

%%% The following command loads the amsrefs package, which will be
%%% used to create the bibliography:
%\usepackage[lite]{amsrefs}

%%% The following command defines the standard names for all of the
%%% special symbols in the AMSfonts package, listed in
%%% http://www.ctan.org/tex-archive/info/symbols/math/symbols.pdf
\usepackage{amssymb}

%%% The following commands allow you to use \Xy-pic to draw
%%% commutative diagrams.  (You can omit the second line if you want
%%% the default style of the nodes to be \textstyle.)
\usepackage[all,cmtip]{xy}
\let\objectstyle=\displaystyle

\usepackage{hyperref}
\usepackage{mathtools}
\usepackage{url}

\usepackage{amsthm}
\let\proof\relax
\let\endproof\relax

\usepackage{lastpage}
\usepackage{fancyhdr}
\fancyfoot[C]{Page \thepage\ of \pageref*{LastPage}}
\pagestyle{fancy}

\usepackage{filemod}

\usepackage{natbib} % for \citep macro
\bibliographystyle{apalike}
\usepackage{xurl}

%%% If you'll be importing any graphics, uncomment the following
%%% line.  (Note: The spelling is correct; the package graphicx.sty is
%%% the updated version of the older graphics.sty.)
% \usepackage{graphicx}

%%% This part of the file (after the \documentclass command,
%%% but before the \begin{document}) is called the ``preamble''.
%%% This is where we put our macro definitions.

%%% Comment out (or delete) any of these that you don't want to use.
\newcommand{\tensor}{\otimes}
\newcommand{\homotopic}{\simeq}
\newcommand{\homeq}{\cong}
\newcommand{\iso}{\approx}

\DeclareMathOperator{\ho}{Ho}
\DeclareMathOperator*{\colim}{colim}

\newcommand{\R}{\mathbb{R}}
\newcommand{\C}{\mathbb{C}}
\newcommand{\Z}{\mathbb{Z}}

\newcommand{\M}{\mathcal{M}}
\newcommand{\W}{\mathcal{W}}

\newcommand{\itilde}{\tilde{\imath}}
\newcommand{\jtilde}{\tilde{\jmath}}
\newcommand{\ihat}{\hat{\imath}}
\newcommand{\jhat}{\hat{\jmath}}

%\newcommand{\updateinfo}[1][\today]{\par\vfill\hfill{\script‌​size\color{gray}Last updated on #1}}


%%%-------------------------------------------------------------------
%%%-------------------------------------------------------------------
%%% The Theorem environments:
%%%
%%%
%%% The following commands set it up so that:
%%% 
%%% All Theorems, Corollaries, Lemmas, Propositions, Definitions,
%%% Remarks, Examples, Notations, and Terminologies  will be numbered
%%% in a single sequence, and the numbering will be within each
%%% section.  Displayed equations will be numbered in the same
%%% sequence. 
%%% 
%%% 
%%% Theorems, Propositions, Lemmas, and Corollaries will have the most
%%% formal typesetting.
%%% 
%%% Definitions will have the next level of formality.
%%% 
%%% Remarks, Examples, Notations, and Terminologies will be the least
%%% formal.
%%% 
%%% Theorem:
%%% \begin{thm}
%%% 
%%% \end{thm}
%%% 
%%% Corollary:
%%% \begin{cor}
%%% 
%%% \end{cor}
%%% 
%%% Lemma:
%%% \begin{lem}
%%% 
%%% \end{lem}
%%% 
%%% Proposition:
%%% \begin{prop}
%%% 
%%% \end{prop}
%%% 
%%% Definition:
%%% \begin{defn}
%%% 
%%% \end{defn}
%%% 
%%% Remark:
%%% \begin{rem}
%%% 
%%% \end{rem}
%%% 
%%% Example:
%%% \begin{ex}
%%% 
%%% \end{ex}
%%% 
%%% Notation:
%%% \begin{notation}
%%% 
%%% \end{notation}
%%% 
%%% Terminology:
%%% \begin{terminology}
%%% 
%%% \end{terminology}
%%% 
%%%       Theorem environments

% The following causes equations to be numbered within sections
\numberwithin{equation}{section}

% We'll use the equation counter for all our theorem environments, so
% that everything will be numbered in the same sequence.

%       Theorem environments

\theoremstyle{plain} %% This is the default, anyway
\newtheorem{theorem}[equation]{Theorem}
%\newtheorem{theorem}{Theorem}[]
\newtheorem{cor}[equation]{Corollary}
\newtheorem{lemma}[equation]{Lemma}
\newtheorem{prop}[equation]{Proposition}
\newtheorem{statement}[equation]{Statement}

\theoremstyle{definition}
\newtheorem{definition}[equation]{Definition}
%\newtheorem{definition}{Definition}
%\newtheorem{proof}{{\bf Proof:}}{\hfill\rule{2mm}{2mm}}

\theoremstyle{remark}
\newtheorem{rem}[equation]{Remark}
\newtheorem{ex}[equation]{Example}
\newtheorem{notation}[equation]{Notation}
\newtheorem{terminology}[equation]{Terminology}

\newenvironment{proof}{{\bf Proof:}}{\hfill\rule{2mm}{2mm}}

%%%-------------------------------------------------------------------
%%%-------------------------------------------------------------------
%%%-------------------------------------------------------------------
%%%-------------------------------------------------------------------
%%%-------------------------------------------------------------------
%%%-------------------------------------------------------------------
%%%-------------------------------------------------------------------
\begin{document}

%%% In the title, use a double backslash "\\" to show a linebreak:
%%% Use one of the following two forms:
%%% \title{Text of the title}
%%% or
%%% \title[Short form for the running head]{Text of the title}
\title{Equivalence of axiom of choice}


%%% If there are multiple authors, they're described one at a time:
%%% First author: \author{} \address{} \curraddr{} \email{} \thanks{}
%%% Second author: \author{} \address{} \curraddr{} \email{} \thanks{}
%%% Third author: \author{} \address{} \curraddr{} \email{} \thanks{}
\author{Sibo WANG}

%%% In the address, show linebreaks with double backslashes:
%\address{}

%%% Current address is optional.
% \curraddr{}

%%% Email address is optional.
% \email{}


%%% If there's a second author:
% \author{}
% \address{}
% \curraddr{}
% \email{}


%%% To have the current date inserted, use \date{\today}:
\date{}

%%% To include an abstract, uncomment the following two lines and type
%%% the abstract in between them:
% \begin{abstract}
% \end{abstract}


\maketitle

%%% To include a table of contents, uncomment the following line:
% \tableofcontents


%%%-------------------------------------------------------------------
%%%-------------------------------------------------------------------
%%% Start the body of the paper here!  E.G., maybe use:
%%% \section{Introduction}
%%% \label{sec:intro}

%%% For a numbered display, use
%%% \begin{equation}
%%%   \label{something}
%%%   The display goes here
%%% \end{equation}
%%% and you can refer to it as \eqref{something}.

%%% For an unnumbered display, use
%%% \begin{equation*}
%%%   The display goes here
%%% \end{equation*}

%%% To import a graphics file, you must have said
%%% \usepackage{graphicx}
%%% in the preamble (i.e., before the \begin{document}).
%%% Putting it into a figure environment enables it to float to the
%%% next page if there isn't enough room for it on the current page.
%%% The \label command must come after the \caption command.
% \begin{figure}[h]
%   \includegraphics{filename}
%   \caption{Some caption}
%   \label{somelabel}
% \end{figure}

\section{Motivation}
%\label{sec:intro}
In this notes, rigorous proofs of some statements, which are equivalent with axiom of choice, will be shown .  

\section{Axiom of choice}
%\label{sec:intro}

\begin{statement}
\textup{(Axiom of choice)} \textit{For every non-empty set $\mathcal{A}$, there exists a choice function $f: \mathcal{P} (\mathcal{A}) \to \mathcal{A}$, and $f(a) \in a$ for each $a \in \mathcal{P} (\mathcal{A})$. } \label{statement:Axiom of choice}
\end{statement}

\section{Hausdorff maximal principle}
The following lemma is a revised version of \citep{Spronk.2019}.
\begin{definition}
Given a  non-empty set $\mathcal{A}$ and $\mathcal{F} \subseteq \mathcal{P}(\mathcal{A})$, we define $\mathcal{B}^*=\{x \in \mathcal{A} \setminus \mathcal{B}: \mathcal{B} \cup \{x\} \in  \mathcal{F} \}$ for every $\mathcal{B} \in \mathcal{F}$. \label{definition:F}
\end{definition}

\begin{definition}
Given a choice function $f$ in \hyperref[statement:Axiom of choice]{Statement \ref*{statement:Axiom of choice}} and $\mathcal{F}$ in \hyperref[definition:F]{Definition \ref*{definition:F}}, we define a function $F: \mathcal{F} \to \mathcal{F}$ \label{definition:B_star}
	\begin{equation*}
	    F(\mathcal{B}) =
	    \begin{cases*}
		      \mathcal{B} \cup {f(\mathcal{B}^*)} & if $\mathcal{B}^* \neq  \varnothing$ \\
		     \mathcal{B}       & otherwise
	    \end{cases*}
	 \end{equation*}
\end{definition}

\begin{definition}
Given a non-empty set $X$, a sub-collection $\mathcal{F} \subseteq \mathcal{P}(X)$ and \hyperref[definition:B_star]{function $F$}: $\mathcal{F} \to \mathcal{F}$, a subset  $\mathcal{T} \subseteq \mathcal{F}$ is called a \textit{tower} of $\mathcal{F}$ if and only if $\mathcal{T}$ has the following three properties: \label{definition:tower}
	\begin{equation*}
	    \varnothing \in \mathcal{T} \tag{3.3.1}\label{definition:3.3.1}
	\end{equation*}
	\begin{equation*}
	   \text{if }\mathcal{A} \in \mathcal{T}, \text{ then } F(\mathcal{A}) \in \mathcal{T} \tag{3.3.2}\label{definition:3.3.2}
	\end{equation*}
	\begin{equation*}
	    \text{if }\mathcal{A} \subset \mathcal{T} \text{and } (\mathcal{A}, \subseteq) \text{ is a chain, then } \bigcup_{a \in \mathcal{A}}a \in \mathcal{T} \tag{3.3.3}\label{definition:3.3.3}
	\end{equation*}
\end{definition}

\begin{lemma}
Given a non-empty set $X$, if a sub-collection $\mathcal{F} \subseteq \mathcal{P}(X)$ has the following two properties: \label{lemma:FF}
	\begin{equation*}
	    \varnothing \in \mathcal{F} \tag{3.4.1}\label{lemma:3.4.1}
	\end{equation*}
	\begin{equation*}
	    \text{if }\mathcal{A} \subset \mathcal{F} \text{and } (\mathcal{A}, \subseteq) \text{ is a chain, then } \bigcup_{a \in \mathcal{A}}a \in \mathcal{F} \tag{3.4.2}\label{lemma:3.4.2}
	\end{equation*}
then there is $\mathcal{M} \in \mathcal{F}$ such that $\mathcal{M} \cup \{x\} \notin \mathcal{F}$  for every $x \in X \setminus \mathcal{M}$. 
\end{lemma}

\begin{proof}
Given $\mathcal{F}$ and a \hyperref[definition:B_star]{function $F$}, we define 
$$\mathcal{T}_0 = \bigcap \{ \mathcal{T} \subseteq  \mathcal{F}: \mathcal{T} \text{ is a tower of }\mathcal{F}\}$$
We can verify $\mathcal{T}_0$ is a tower by mathematical induction. \\
Next our target is to prove $(\mathcal{T}_0, \subseteq)$ is a totally ordered set.     \\
We find an element $C \in \mathcal{T}_0$ such that $C$ is a comparable element, i.e. for each $t \in \mathcal{T}_0 \setminus \{{C}\}$, $t \subset {C}$ or $t \supset {C}$. Indeed $C$ is well defined, because $\varnothing \in \mathcal{T}_0$. From ${C}$, we define 
$$\mathcal{T}_C = \{ t \in \mathcal{T}_0: t \subset C \} \bigcup \{ C \} \bigcup \{ F(C) \} \bigcup \{ t \in \mathcal{T}_0: F(C) \subset t \}$$
It is trivial to see that $\mathcal{T}_C \subseteq \mathcal{T}_0$. \\
We can prove $\mathcal{T}_C$ is a tower by \hyperref[definition:tower]{definition  \ref*{definition:tower}}.  \\
\hyperref[definition:3.3.1]{(\ref*{definition:3.3.1})}: $\varnothing \subseteq C$, so $\varnothing \in \mathcal{T}_C$. \\  
\hyperref[definition:3.3.2]{(\ref*{definition:3.3.2})}: For each $x \in \mathcal{T}_C$, $x \in \mathcal{T}_0$ and $F(x) \in \mathcal{T}_0$, because $\mathcal{T}_C \subseteq \mathcal{T}_0$. If $x \subset C$, then $F(x) \subseteq C$, thus $F(x) \in \mathcal{T}_C$; if $x \supset C $, then $F(C) \subseteq F(x)$, thus $F(x) \in \mathcal{T}_C$. \\
\hyperref[definition:3.3.3]{(\ref*{definition:3.3.3})}: If there is a chain $(\mathcal{D}, \subset)$ in $\mathcal{T}_C$ and $e = \bigcup_{d \in \mathcal{D}}d $, then $e \in \mathcal{T}_0$, because $\mathcal{T}_C \subseteq \mathcal{T}_0$. If $d \subseteq C$ for each $d \in \mathcal{D}$, then $e \subseteq C$, thus $e \in \mathcal{T}_C$; if $C \subseteq d$ for some $d \in \mathcal{D}$, then $C \subseteq e$, then $F(C) \subseteq e$, thus $e \in \mathcal{T}_C$. \\
We can get $\mathcal{T}_C = \mathcal{T}_0$ because $\mathcal{T}_0$ is the minimal tower.  \\
Define \\
$$U = \{ C \in \mathcal{T}_0: C \text{ is a comparable element}\}$$
It is trivial to see that $U \subseteq \mathcal{T}_0$ and $(U, \subset)$ is a well ordered set. \\
We can prove $U$ is a tower by \hyperref[definition:tower]{definition  \ref*{definition:tower}}.  \\
\hyperref[definition:3.3.1]{(\ref*{definition:3.3.1})}: $\varnothing$ is subset of any set. \\  
\hyperref[definition:3.3.2]{(\ref*{definition:3.3.2})}: From definition of $\mathcal{T}_C$, we know $F(x)$ is a comparable element in $\mathcal{T}_C$ for any comparable element $x$ in $\mathcal{T}_C$, so $F(x)$ is a comparable element in $\mathcal{T}_0$ as well. \\
\hyperref[definition:3.3.3]{(\ref*{definition:3.3.3})}: Given any chain $(\mathcal{D}, \subset)$ in $U$, $e = \bigcup_{d \in \mathcal{D}}d$, $e \in \{d: d \in \mathcal{D}\} \cup \{ F(d): d \in \mathcal{D} \} \subseteq U$.\\
Therefore, $U$ is a tower. We finish proving $U = \mathcal{T}_0$ and $(\mathcal{T}_0, \subset)$ is a totally ordered set.  \\
Define  
$$\mathcal{M} = \bigcup_{t \in \mathcal{T}_0}t $$
Because $(\mathcal{T}_0, \subset)$ is a chain, $\mathcal{M} \in \mathcal{T}_0$ by definition of $\mathcal{T}_0$. Suppose $\mathcal{M}^*\neq \varnothing$, then $\mathcal{M} \subset F(\mathcal{M}) \in \mathcal{T}_0$, which would violate the definition of $\mathcal{M}$. Therefore, $\mathcal{M}^*= \varnothing$ and $\mathcal{M}$ is what we want. 
\end{proof}

\begin{statement}
\label{statement:Hausdorff maximal principle}
\textup{(Hausdorff maximal principle)} \textit{For every partially ordered set $(\mathcal{S}, \leq)$, there is a chain $\mathcal{M}$, such that $\mathcal{M} \cup \{s\}$ is not a chain for every $s \in \mathcal{S} \setminus \mathcal{M}$.}
\end{statement}

\begin{theorem}
\hyperref[statement:Axiom of choice]{Axiom of choice} implies \hyperref[statement:Hausdorff maximal principle]{Hausdorff maximal principle}.
\end{theorem}

\begin{proof}
Given a partially ordered set $(\mathcal{S}, \subseteq)$ and let $\mathcal{F}$ denote the set of all chains in $\mathcal{S}$. We can verify that $\mathcal{F}$ has two properties in \hyperref[lemma:FF]{Lemma \ref*{lemma:FF}}: \\
\hyperref[lemma:3.4.1]{(\ref*{lemma:3.4.1})}: $\varnothing$ is a chain. \\  
\hyperref[lemma:3.4.2]{(\ref*{lemma:3.4.2})}: Given any chain $(\mathcal{D}, \subseteq)$ in $\mathcal{F}$ and $e = \bigcup_{d \in \mathcal{D}}d$. $e \in \mathcal{F}$ because $e$ is also a chain.\\
Therefore, the chain $\mathcal{M}$ can be found like \hyperref[lemma:FF]{Lemma \ref*{lemma:FF}}. 
\end{proof}

\section{Zorn's Lemma}

\begin{statement}
\label{statement:Zorn's Lemma}
\textup{(Zorn's Lemma)} \textit{Given a partially ordered set $(\mathcal{S}, \subseteq)$, if every chain of $(\mathcal{S}, \subseteq)$ has an upper bound, then there is a maximal element $m$ for $\mathcal{S}$. }
\end{statement}

\begin{theorem}
\hyperref[statement:Hausdorff maximal principle]{Hausdorff maximal principle} implies \hyperref[statement:Zorn's Lemma]{Zorn's Lemma}.
\end{theorem}

\begin{proof}
Let $(\mathcal{S}, \subseteq)$ be the partially ordered set, then we can get a maximal chain  $\mathcal{M}$ in $(\mathcal{S}, \subseteq)$ by \hyperref[statement:Hausdorff maximal principle]{Hausdorff maximal principle}. Let $m$ be an upper bound for $\mathcal{M}$. Then $\mathcal{M} \cup \{m\} = \mathcal{M}$. Therefore, $m$ is a maximal element for $\mathcal{S}$. 
\end{proof}

\section{Well-ordering principle}
To be continued

\section{Tychonoff's theorem}
To be continued

%%% -------------------------------------------------------------------
%%% -------------------------------------------------------------------
%%% This is where we create the bibliography.

%\begin{bibdiv}
%  	\begin{biblist}
%  	
%%\url(http://www.math.uwaterloo.ca/~nspronk/math453/AofC.pdf)\\
%%\url(http://www.math.toronto.edu/ivan/mat327/docs/notes/11-choice.pdf)
%
%
%	\end{biblist}
%\end{bibdiv}

\bibliography{mybib.bib}
%\printbibliography


\end{document}